% template d'article LaTeX créé par Max De Wilde (STIC - ULB)
% contact : madewild@ulb.ac.be

\documentclass[a4paper,11pt]{article} % ce document est un article sur une feuille A4, police taille 11

\usepackage[utf8]{inputenc} % encodé en utf-8
\usepackage[T1]{fontenc} % compatible avec les accents

\usepackage[round]{natbib} % gestion des citations
\usepackage[french]{babel} % rédigé en français
\usepackage[hyphens]{url} % formatte les liens en autorisant la césure au niveau des traits d'union
\usepackage[pdftex,urlcolor=black,colorlinks=true,linkcolor=black,citecolor=black]{hyperref} % liens cliquables mais non colorés
\usepackage[top=3cm,bottom=4cm]{geometry} % gère les marges
\usepackage{graphicx} % gestion des images
\usepackage{array} % gestion des tableaux
\usepackage{csquotes} % gestion des guillemets
\usepackage{fourier} % utilise une autre police que celle par défaut (Computer Modern)

% insérez ici d'autres extensions avec la commande \usepackage[options]{nom de l'extension}

\title{Projet ASI : La convergence des transitions écologique et numérique : état des lieux et perspectives wallonnes} % le titre de l'article
\author{Sylviane Van Rymenant} % vos prénom et nom
\date{} % pas de date

\begin{document} % début du corps du texte
\maketitle % affiche le titre, l'auteur et la date

\section{Introduction} % section 1
L’avenir sera numérique. Cette ambition se retrouve dans la plupart des programmes ou déclarations des instances dirigeantes internationale. L’Union européenne par exemple a publié (un truc) exposant ses objectifs pour un avenir numérique européen. La montée en croissance du numérique crée des emplois, des opportunités, de la compétitivité par rapport aux autres pays, plus de confort, des nouvelles connaissances, etc. Les technologies sont incontournables et elles sont au cœur de beaucoup de projets et d’ambitions publiques ou privée, pour créer un monde de demain connecté, optimisé et « intelligent ». 
\\
\\L’avenir sera « vert » ! Le réchauffement climatique ne fait désormais plus de doutes, cela fait des décennies que les scientifiques pointent du doigt les hausses alarmantes de la température planétaire et des effets désastreux qui en découlent ou en découleront. Des mesures fortes et contraignantes doivent être prises et doivent être prises rapidement. La sensibilisation à cet enjeu crucial a pris beaucoup de terrain pour être désormais incontournable. Les citoyens sont conscients de la crise climatique que nous vivons, les politiques aussi et de nombreuses actions sont mises en place à travers le monde. Les marches pour le climat, des initiatives citoyennes, des projets à grande ou petites échelles ont vu le jour en même temps que les gouvernements commençaient à mettre en place de vraies mesures pour contribuer à la transition écologique. 
\\
\\
Ces deux transitions, ces deux enjeux fondamentaux et incontournables sont donc voués à cohabiter. Mais cela est-il réellement possible ? Le problème avec ces deux enjeux, c’est que le développement de l’un d’eux (numérique) aura un impact négatif sur l’autre (écologie) s’il n’est pas adapté. En effet, l’impact des Technologies de l’Information et de la Communication sur l’environnement a déjà été pointé du doigts par différents chercheurs de nombreuses fois. Le sujet n’est pas nouveau, beaucoup d’articles existent à ce sujet. Mais la question que l’on peut se poser est la suivante : Qu’est-ce qui a réellement été mis en place pour rendre les TIC plus vertes ? Car la théorie est une chose et la pratique une autre. De la même manière, il est intéressant de faire un état des lieux des technologies numériques susceptibles de contribuer à la transition écologique.
En effet, le paradigme de la « dématérialisation » a échoué à tenir ses promesses : cette « dématérialisation » n’en n’est finalement pas réellement une, ou plutôt, le numérique n’entraîne finalement pas une diminution des ressources consommées. Par ailleurs, la principale source de pollution due aux TIC, est l’extraction des minéraux nécessaires à la fabrication des appareils électroniques. De plus, une action numérique a en fait un cout écologique, le stockage dans des centres de données consomme, regarder une série sur Netflix consomme beaucoup, faire un achat sur Amazone, c’est évident entraîne toutes la logistiques qui va avec pour amener notre achat à bon port. Sans compter sur l’impact indirecte des TIC, à savoir leur impact sur la consommation des citoyens, et le modèle qu’elle promulgue : celui de la surconsommation. 
Tous ces aspects font douté certains de la compatibilité entre les domaines. Cependant, les TICs peuvent en réalité servir d’outil à la transition écologique.
\\
\\
Deux enjeux sont donc au cœur de la transition numérique : Premièrement, celui de réduire l’impact sur l’environnement des technologies et du numérique. Cela passe par la réutilisation des matériaux nécessaires à leur fabrication, à l’optimisation des (centres de données, des sites, je sais pas à VOIR SE RENSEIGNER). Deuxièmement, utiliser les innovations technologiques au service des ambitions climatiques. En effet, les innovations technologiques sont nombreuses, plurielles et multidisciplinaire. Rares sont les secteurs qui échappent à la numérisation, qui trouve de nombreux moyens d’optimiser le travail, d’améliorer les conditions de vie, de travail, d’automatiser certains processus, etc. Le numérique peut donc également servir les objectifs écologique et durables, de multipe manière que nous étudions dans ce travail.
\\
\\
De nombreux travaux ont déjà été réalisé concernant la convergence de ces deux enjeux. Citons notamment (………), certains évoquant des pespectives plutôt positives, tandis que d’autres, plus pessimistes soulignent plutôt l’incompatibilité entre les deux, en tout cas telle qu’elles sont maintenant.
\\
\\
Si de nombreux travaux internationaux et français existent, je n’en ai trouvé aucun faisant été de la situation belge (et plus particulièrement wallonne). 
La Région Wallonne, comme bien d’autres régions européennes met en avant non seulement ses ambitions numériques, trouvant leur point d’appui dans la nouvelle plateforme DigitalWallonia, mais également écologiques, les deux étant souvent liés ou en tout cas pas exclu l’un de l’autre. Des initiatives wallonnes ont pourtant bien été mises en place, les enjeux écologiques et numériques se retrouvant partout dans les projets et déclaration des gouvernements. Le site DigitalWallonie a pour but de centraliser les initiatives wallonnes, de faire un état des lieux, de servir de site de référence, de ressource, de plateforme de partage d’expériences wallonnes, etc. Certains de leur projet sont directement lié à la « vertification » du numérique, qu’ils mettent notamment en lien avec les objectifs du Green Deal européen.


\section{État de l'art} % section 2
Le texte de l'état de l'art selon

\begin{figure}[h] % insère une figure ici (h = "here")
  \centering % centre la figure
  \includegraphics[scale=1]{image} % insère une image en taille réelle
  % l'extension n'est pas précisée pour éviter des problèmes de compilation
  % plus d'info ici : http://fr.wikibooks.org/wiki/LaTeX/Inclure_des_images
  \caption{Logo du MaSTIC} % nom de l'image
\end{figure}

\section{Analyse} % section 3

\subsection{Ambitions numérique et écologique européennes et wallonnes}
\\
\\
\subsubsection{l'Union européenne}
\\

L’Union Européenne a fait de la transition écologique ainsi que de la transition numérique priorités. Le Green Deal Pacte Vert cristallise tout ces objectifs : L’Europe veut devenir « le premier continent neutre pour le climat ». Ce projet ambitieux répond au défi pressant que nous soumet le changement climatique. Ce Pacte débouchera sur des actions concrètes et des changements de politiques répondant à cet enjeux. Et l’UE entend utiliser les TIC pour arriver aux buts qu’elle s’est fixer. (RAJOUTER REFERENCE ET INFORMATIONS SUR LE GREEN DEAL) Ce mélange repose sur plusieurs objectifs : 
\\ 
\\1) le lancement d’une nouvelle stratégie industrielle de l’UE 
\\ 
\\2)le rencorcement des capacités de l’UE en mantière de prévision et de gestion des catastrophes environnementales (citons notamment l’Intitiative « Destination Terre » qui est un modèle numérique de la Terre de haute précision) 
\\
\\3)Soutien de l’économie circulaire réseaux haut débit, centres de données et équipement informatique,etc. En introduisant notamment de nouveaux « passeports produits » permettant de suivre le trajet parcourus par un produit. 
\\ 
\\4)Lancement d’une initiative d’économie circulaire pour le matériel éléctronique (allonger la durée de vie des appareils électroniques), 
\\
\\5)Rendre les centres de données et les infrastructures TIC climatiquement neutres d’ici 2030. 
\\
\\6)Tirer profit de l’intelligence artificielle, de la 5G, de l’informatique en usage et du traitement des données à la périphérie, ainsi que de l’internet des objets. 
\\ 
\\7)Encourager les transports automatisés et connectés (systèmes intelligents) \\ \\
\\
\\8)Rendre les marchés publics plus durables.
\\
\\
Concernant les objectifs numériques de l’Europe, le sujet technologique est par bien des aspects important pour l’Europe, c’est une question stratégique pour être compétitif sur le marché international. Par ailleurs, étant donné l’étendue des domaines d’application du numérique et de ce qu’il permet (diminution des risques, augmentation des rendements, augmentation du confort de vie), c’est pour l’Europe, comme pour le monde, un outil indipensable à la construction du monde de demain (du  moins c’est ce que les instances dirigeantes disent). Plusieurs « feuilles de route » ont été réalisées réccemment, en 2020-2021, par la Commission européenne, comme par exemple « Soutenir la transition écologique – Façonner l’avenir numérique de l’Europe », « Une boussole numérique pour 2030 : l’Europe balise la décennie numérique.
\\La Commission européenne l'affirme : les transformations écologiques et numériques sont liées et interdépendantes. Il est nécessaire de réaliser une coconstruction du monde de demain avec ces deux éléments. L'ère du numérique ne peut ignorer la dimensions écologique, et les mesures "vertes" ne sauront passer à côté de la puissante "force transformatrice" que représente la transition numérique, comprable, en ampleur et en impact à la Révolution industrielle.
\\ Cependant, les évolutions technologiques et l'omniprésence du numérique dans nos vies comprend aussi des risques, dont les citoyens sont de plus en plus consceints (notamment le risque concernant l'exposition de leurs données personnels, ou encore le temps excessif passés sur les écrans, ou bien sûr, l'impact environnemental)
\\La Commission a donc pour objectif de créer un avenir technologique qui lui correspond, qui intégrera ses propres valeurs sans être soumis à des règles internationales qui ne respecteraient pas les droits des citoyens européens (l'adoption du RGPD est représentatif de la direction que l'Europe veut donner à sa stratégie numérique). De cette manière, le monde numérique européen de demain sera vert, pour correspondre aux ambitions du Pacte Vert de l'Europe. Les objectifs technologiques de l'Europe ne pourront pas se réaliser sans prendre en compte cette valeur désormais fondamentale de l'Europe de respect et de protection de l'environnement.
\\
\\
\subsubsection{La Région wallonne}
\\
\\ (je rajoute du textes pour tester, ça fonctionne)\)
\\
\\
Comme l'Union Européenne, la Wallonie a également établi sa stratégie numérique pour demain. Et comme l'Union Européenne, la transition écologique est un enjeu fondamental et transversal pour la région.
\\
\\Concernant la transition numérique, la stratégie wallonne a pris corps au travers du site "DigitalWallonia.be". Cette plateforme

\subsection{Green IT ou informatique durable}
\\
\\
\\

\subsection{IT for Green}
Le texte de l'analyse...\footnote{\url{http://mastic.ulb.ac.be}}

\subsubsection{les TIC au service du développement durable}
\\
\\
Si l’économie circulaire a droit à son propre sous-chapitre c’est que c’est un domaine d’application pour les tic qui revient très régulièrement quand on parle de l’interconnexion entre tic et transition écologique. Le numérique (et l’innovation) est d’ailleurs présenté comme un des 9 leviers d’action dans le plan de développement de l’économie circulaire wallon de (2020 ?).
\\
\\
L’interconnection entre numérique et développement de l’économie circulaire (plus durable) est souvent mises en avant. La Wallonie a d’ailleurs crée un projet, appelé « Circular Wallonia ». Dans un Rapport stratégie de déploiement de l’économie circulaire, la Région Wallonne présente ses ambitions (sociale, écologique et économique) liées au développement de l’économie circulaire sur son territoire. Le but est notamment d’être plus indépendant vis-à-a vis du reste du monde, notamment en situation de crise (comme le covid) où l’on se retrouve finalement démuni devant les pénuries de ressources venant de l’étranger. Le rapport explique que numérique et économie circulaire sont fortement liés, pour deux raisons : la première est que les outils que proposent les tic peuvent être d’excellents outils permettrant d’optimiser et accélérer les objectifs de « circularité ». La seconde est, étant donné le fort impact sur l’environnement, de la fabrication des appareils électroniques, l’enjeu d’intégrer les tic elles-mêmes, en tant qu’objets dans ce processus pour maximiser la durée de vie des appareils. Dans ce rapport, on apprend également que déjà maintenant, l’économie circulaire est un secteur d’emplois important pour le secteur des technologies digitales (en fournissant environ 6000 emplois), l’objectif étant que ce nombre s’accroisse encore. 
\\
\\
« A travers Digital Wallonia, la Wallonie veillera à mieux identifier les technolgies numériques qui apportent de réelles solutions à l’économie circulaire (…) La thématique « économie circulaire » sera identifiée par un prochain appel à projets « Smart Region » de la stratégie Digital Wallonia. Plusieurs projets numériques ont déjà été développés par les pouvoirs locaux sur le territoire wallon : application mobiles, éclairage intelligent, plateforme participatives, capteurs de mesures de la qualité de l’air, etc. Un appel à projet de ce type stimulera le développement de projets numériques territoriaux favorisant le déploiement de l’économie circulaire ».

\\
\subsubsection{Big Data}
\\
\\
L’utilité et l’importance de l’utilisation des « Big Data » dans la crise climatique que nous traversons a déjà été démontrée. Premièrement, ce sont les données concernant de l’évolution de la température de la Terre, etc. qui ont permis d’alerter la communité scientifique du phénomène de réchauffement climatique. Sans les données issues de l’observation terrestre, de ses phénomènes et des changements qui la parcourent, comment voir l’apparition d’une nouvelle menace ?
\\
\\
\subsubsection{Blockchain}

\subsubsection{Smart cities}
\\combinaison de différentes technologies au service de l'amélioraiton des conditions de vie en ville.
\\
\\L’auteur (2017 Comment tansition numérique et transition écologique s’interconnectent elles) évoqué dans la section sur la blockchain, évoque également les smart cities comme une des solution apportées par les tics pour un avenir plus durable. Les Smarts cities, dures à définir, peuvent se déployer sur bien des aspects. L’idée derrière le concept est, comme son nom l’indique, de développer des « villes intelligentes » et connectées, qui utiliseraient des grandes quantités de données pour optimiser la vie en ville. Les technologies ont donc un grand rôle à jouer pour mettre en œuvre ce concept. Des projets de Smart cities ont déjà vu le jour dans beaucoup de villes européenne, sous de multiples formes. L’idée n’est évidemment de créer directement des villes complètement intelligentes et connectées, mais certains aspects ou problèmes de villes sont amélioré  grâce à des technologiques. On peut réduire les embouiteillages, faire en sortes que ceci  cela (DONNER DES EXEMPLES !!!).
\\
Etant donné que la population des villes se densifie et que la pollution de l’air y est la plus pregnante et inquiétante, ce modèle de smart cities a aussi son rôle à jouer d’un point de vue environnemental. Réduire la taille et l’ampleur des embouiteillages, n’est par exemple, pas qu’une question d’efficacité, c’est aussi une question environnemental.
Certaines craintes de ce modèle de Smart cities ont néanmoins vu le jour. Beaucoup pointent du doigt la trop grande dépendance aux technologies.

\\
\\
\subsubsection{La Communication}
\\
\\



\begin{table}[h] % insère un tableau ici
  \centering % centre le tableau
  \begin{tabular}{|l|c|r|} % insère un tableau avec 3 colonnes centrées à gauche (l), au centre (c) et à droite (r)
    \hline % ligne horizontale
    STIC3 & STIC4 & STIC5 \\ % contenus des cellules séparés par des & et \\ en fin de ligne
    \hline
    STIC3 & STIC4I & STIC5I \\
    STIC3 & STIC4C & STIC5C \\
    \hline
  \end{tabular}
  \caption{Finalités du MaSTIC}
\end{table}

\section{Conclusion} % section 4
\\
\\ Dans ce travail, on a parcouru les textes officiels du gouvernement wallon, et de la Commission européenne pour voir quelles solutions ils envisageaient pour l’avenir pour faire coexister et collaborer les acteurs et les enjeux de ces deux domaines : le numérique et le développement durable. 
\\
Si de nombreux objectifs sont été inscrits sur le papier, proposés, développé, reste encore à voir ce que cela donnera dans la pratique.
Un aspect manque dans ce travail : ce sont les perspectives, les projets, les solutions, proposées non pas par des instances dirigeantes mais par des citoyens.

\enquote{Une bonne conclusion est une conclusion finale.} 

\bibliographystyle{plainnat-fr} % paramètre l'affichage de la bibliographie
\bibliography{biblio} % indique que la bibliographie se trouve dans le fichier biblio.bib

\end{document} % fin du corps du texte